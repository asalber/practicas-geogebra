\documentclass[a4paper]{article}
\usepackage{svn-multi}
% Version control information:
\svnidlong
{$HeadURL: https://practicas-derive.googlecode.com/svn/trunk/matrices.tex $}
{$LastChangedDate: 2008-11-13 12:37:02 +0100 (jue, 13 nov 2008) $}
{$LastChangedRevision: 3 $}
{$LastChangedBy: asalber $}
\svnid{$Id: matrices.tex 3 2008-11-13 11:37:02Z asalber $}
\pdfinfo{/CreationDate (D:\svnpdfdate)}
\svnRegisterAuthor{alf}{Alfredo Sánchez Alberca}

\usepackage[spanish]{babel}
\usepackage[utf8x]{inputenc}
\usepackage{amsmath}
\usepackage{macros}
\usepackage[dvips]{graphicx}
\usepackage{enumitem}
\usepackage{subfigure}
\usepackage[small,bf]{caption2}
\usepackage[top=3cm, bottom=3cm, left=2.54cm, right=2.54cm]{geometry}
\usepackage{fancyhdr}
\pagestyle{fancy}

\lhead{\textsc{Universidad San Pablo CEU}} \rhead{\textsl{\textsf{Departamento de Métodos Cuantitativos}}}
\renewcommand{\headrulewidth}{0pt}
\renewcommand{\floatpagefraction}{.8}
\renewcommand{\textfraction}{.1}
\setcaptionwidth{\textwidth} \addtolength{\captionwidth}{-40pt}
\captionstyle{indent} \setlength\captionindent{\parindent}

\makeatletter
\let\savees@listquot\es@listquot
\def\es@listquot{\protect\savees@listquot}
\makeatletter

\begin{document}
\practica{Práctica de Álgebra con DERIVE}{Matrices}

\bigskip
\section*{Fundamentos Teóricos}
Las matrices son, junto a los vectores, las otras estructuras elementales del álgebra.
En esta práctica se introduce el concepto de matriz y se muestran las operaciones básicas 
del cálculo matricial. En prácticas posteriores se mostrarán algunas de las múltiples
aplicaciones de las matrices. 

\subsection*{Matrices}
Una matriz es un conjunto de números $a_{ij}$, $i=1,\ldots,m$, $j=1,\ldots,n$ ordenados 
en $m$ filas y $n$ columnas, de la forma
\[
\left(
\begin{array}{cccc}
 a_{11} & a_{12} & \cdots & a_{1n} \\
 a_{21} & a_{22} & \cdots & a_{2n} \\
 \vdots & \vdots & \ddots & \vdots \\
 a_{m1} & a_{m2} & \cdots & a_{mn} \\
\end{array}
\right).
\]

Dependiendo del conjunto al que pertenezcan los números que constituyen la matriz, hablaremos
de matrices, reales, complejas, etc.

El número de filas y el número de columnas de una matriz determinan \emph{dimensión}. Así, una matriz de $m$ filas y $n$ columnas tendrá dimensión $m\times n$. Cuando $n=m$ se dice que la matriz tiene \emph{orden} $n$.

En álgebra, una matriz suele interpretarse como un conjunto de vectores cuyas componentes se ponen en forma de columnas, aunque a veces también se colocan como filas, y por eso se habla de vectores fila o vectores columna.

\subsection*{Tipos de matrices}
Algunos tipos de matrices especiales son:
%\subsubsection*{Según la disposición de sus elementos}
\begin{description}
\item [Matriz nula] Es aquella cuyos elementos valen todos 0.
\item [Matriz cuadrada] Es aquella matriz que tiene el mismo número de filas que columnas.
\item [Matriz diagonal] Es una matriz cuadrada cuyos elementos son todos nulos excepto los
de la diagonal principal, es decir $a_{ij}=0$ si $i\neq j$.
\item [Matriz identidad] Es una matriz diagonal cuyos elementos de la diagonal valen 1, es decir
$a_{ij}=0$ si $i\neq j$ y $a_{ij}=1$ si $i=j$. Se representa como $I$ o $I_n$ si se quiere indicar el orden.
\item [Matriz triangular] Es aquella matriz en la que todos los elementos por debajo de la diagonal principal
son nulos, es decir, $a_{ij}=0$ $\forall i>j$ (matriz triangular superior). O bien es aquella matriz en la que todos los elementos por encima de la diagonal principal
son nulos, es decir, $a_{ij}=0$ $\forall i<j$ (matriz triangular inferior).
\item [Matriz traspuesta] Si $A$ es una matriz de dimensión $m\times n$, diremos que la matriz $B$ de dimensión $n\times m$ es la \emph{matriz traspuesta} de $A$, si las filas de $A$ son las columnas de $B$, es decir, si $a_{ij}=b_{ji}$, $i=1,\ldots,m$, $j=1,\ldots,n$. La matriz traspuesta de $A$ se representa como $A^\textrm{T}$ o bien $A'$.
\item [Matriz simétrica] Es aquella matriz que coincide con su traspuesta, es decir, $a_{ij}=a_{ji}$, $i=1,\ldots,n$, $j=1,\ldots,n$. Resulta evidente que una matriz simétrica es cuadrada.
\item [Matriz antisimétrica] Es aquella matriz que coincide con su traspuesta cambiada de signo, es decir, $a_{ij}=-a_{ji}$, $i=1,\ldots,n$, $j=1,\ldots,n$. También resulta evidente que una matriz antisimétrica es cuadrada. 
\end{description}

\subsection*{Traza y determinante de una matriz cuadrada}
Dada una matriz $A$ de orden $n$, se llama \emph{traza} a la suma de los elementos de la diagonal principal
\[
tr(A)=\sum_{i=1}^{n}a_{ii}.
\]
Y se llama \emph{determinante} de $A$, y se nota $|A|$, a la suma de los $n!$ productos signados de $n$ factores que se obtienen considerando los elementos de la matriz, de forma que cada producto contenga un elemento y sólo uno de cada fila y cada columna de $A$.

\subsection*{Rango de una matriz}
Se llama rango de una matriz $A$ al orden de la submatriz cuadrada de $A$ de mayor orden cuyo determinante no se anule, y se denota como $\textrm{rg}(A)$. Según esto, resulta evidente que el rango de una matriz no puede ser mayor que su número de filas o columnas.

\subsection*{Operaciones matrices}

\begin{description}
\item [Suma de matrices]
Dadas dos matrices
\[A=
\left(
\begin{array}{cccc}
 a_{11} & a_{12} & \cdots & a_{1n} \\
 a_{21} & a_{22} & \cdots & a_{2n} \\
 \vdots & \vdots & \ddots & \vdots \\
 a_{m1} & a_{m2} & \cdots & a_{mn} \\
\end{array}
\right)
\quad \textrm{y} \quad
B=\left(
\begin{array}{cccc}
 b_{11} & b_{12} & \cdots & b_{1n} \\
 b_{21} & b_{22} & \cdots & b_{2n} \\
 \vdots & \vdots & \ddots & \vdots \\
 b_{mn} & b_{m2} & \cdots & b_{mn} \\
\end{array}
\right)
\]
ambas con la misma dimensión $m\times n$, se define su suma como
\[
A+B=
\left(
\begin{array}{cccc}
 a_{11}+b_{11} & a_{12}+b_{12} & \cdots & a_{1n}+b_{1n} \\
 a_{21}+b_{21} & a_{22}+b_{22} & \cdots & a_{2n}+b_{2n} \\
 \vdots & \vdots & \ddots & \vdots \\
 a_{m1}+b_{m1} & a_{m2}+b_{m2} & \cdots & a_{mn}+b_{m2} \\
\end{array}
\right).
\]
Como se ve la suma de matrices se realiza elemento a elemento, por lo que es necesario que las matrices que se suman tengan la misma dimensión.

\item [Producto de una matriz por un escalar] Dada una matriz $A$ y un escalar $c$, se define su producto como  
\[cA=c
\left(
\begin{array}{cccc}
 a_{11} & a_{12} & \cdots & a_{1n} \\
 a_{21} & a_{22} & \cdots & a_{2n} \\
 \vdots & \vdots & \ddots & \vdots \\
 a_{m1} & a_{m2} & \cdots & a_{mn} \\
\end{array}
\right)
=
\left(
\begin{array}{cccc}
 ca_{11} & ca_{12} & \cdots & ca_{1n} \\
 ca_{21} & ca_{22} & \cdots & ca_{2n} \\
 \vdots & \vdots & \ddots & \vdots \\
 ca_{m1} & ca_{m2} & \cdots & ca_{mn} \\
\end{array}
\right)
\]
Como se ve la matriz resultante se obtiene multiplicando cada elemento de $A$ por el escalar $c$, y en consecuencia, la matriz resultante tiene la misma dimensión que $A$.


\item [Producto de matrices]
Dadas dos matrices
\[A=
\left(
\begin{array}{cccc}
 a_{11} & a_{12} & \cdots & a_{1n} \\
 a_{21} & a_{22} & \cdots & a_{2n} \\
 \vdots & \vdots & \ddots & \vdots \\
 a_{m1} & a_{m2} & \cdots & a_{mn} \\
\end{array}
\right)
\quad \textrm{y} \quad
B=\left(
\begin{array}{cccc}
 b_{11} & b_{12} & \cdots & b_{1p} \\
 b_{21} & b_{22} & \cdots & b_{2p} \\
 \vdots & \vdots & \ddots & \vdots \\
 b_{n1} & b_{n2} & \cdots & b_{np} \\
\end{array}
\right)
\]
con dimensiones $m\times n$ y $n\times p$ respectivamente, se define su producto como
\[
AB=
\left(
\begin{array}{ccc}
 a_{11}b_{11}+\cdots+a_{1n}b_{n1} & \cdots & a_{11}b_{1p}+\cdots+a_{1n}b_{np} \\
 a_{21}b_{11}+\cdots+a_{2n}b_{n1} & \cdots & a_{21}b_{1p}+\cdots+a_{2n}b_{np}\\
 \vdots & \ddots & \vdots \\
 a_{m1}b_{11}+\cdots+a_{mn}b_{n1} & \cdots & a_{m1}b_{1p}+\cdots+a_{mn}b_{np} \\
\end{array}
\right).
\]
En la matriz resultante, el elemento que ocupa la fila $i$ y la columna $j$ se obtiene como el producto escalar de el vector de la fila $i$ de $A$ por el vector de la columna $j$ de $B$. De este modo, para poder realizar el producto es necesario que el número columnas de $A$ sea igual al número de filas de $B$, y la matriz resultante tiene dimensión $m\times p$. 

Según esta definición de producto resulta obvio que el producto de matrices no cumple la propiedad conmutativa.
\end{description}

\subsubsection*{Inversa de una matriz}
Dada una matriz cuadrada $A$ de orden $n$, si existe otra matriz $B$ también de orden $n$, tal que $AB=I$, entonces se dice que $A$ es invertible, y a la matriz $B$ se le llama inversa de $A$. Si $A$ es invertible, su inversa es única y se suele notar como $A^{-1}$. Además se cumple $AA^{-1}=A^{-1}A=I$.

Según esta definición, las matrices que no sean cuadradas no tienen inversa. Por otro lado, en el caso de que $A$ sea una matriz cuadrada, una condición necesaria y suficiente para que exista su inversa es que su determinante no se anule.

\section*{Ejercicios Prácticos}
\begin{enumerate}[leftmargin=*]
\item Dadas las matrices
\[
A=\left(
\begin{array}{rrr}
 2 & 1 & 0 \\
 1 & 2 & 1 \\
 3 & 3 & 1 \\
\end{array}
\right),
\quad
B=\left(
\begin{array}{rrr}
 -1 & 2 & 0 \\
  2 & 0 & 1 \\
  1 & 3 & 2 \\
\end{array}
\right),
\quad
C=\left(
\begin{array}{rrr}
 2 &  x &    1 \\
 3 &  2 & 2x^2 \\
 6 & -1 &    y \\
\end{array}
\right),
\]
se pide:

\begin{enumerate}
\item Definir dichas matrices en Derive (menú \texttt{Author->Matrix}).
\item Calcular las siguientes matrices:
\begin{enumerate}
\item $A+B$.
\item $5A+6B-C$. 
\end{enumerate}
\item Calcular los siguientes productos:
\begin{enumerate}
\item $A\cdot B$.
\item $B\cdot 5C$.
\item $A\cdot(2B-C)$.
\end{enumerate}

\item Calcular la traza de $A$, $B$ y $C$ (utilizar la función \texttt{Trace}).

\item Calcular el determinante de $A$, $B$ y $C$ (utilizar la función \texttt{Det}).

\item Calcular el rango de $A$, $B$ y $C$ (cargar el fichero \texttt{vertor.mth} mediante el menú \texttt{File->Load->Math} y utilizar la función \texttt{Rank}).

\item Calcular las siguientes matrices inversas si se puede:
\begin{enumerate}
\item $A^{-1}$.
\item $B^{-1}$.
\item $C^{-1}$.
\item $(2B+3C)^{-1}$.
\end{enumerate}

\item Calcular las siguientes matrices traspuestas:
\begin{enumerate}
\item $A'$. Comprobar que $A+A'$ es una matriz simétrica. ¿Qué ocurre con $A-A'$?¿Y con $A'A$?
\item $(A-2C+B)'$.
\end{enumerate}
\end{enumerate}
\end{enumerate}

\section*{Problemas}
\begin{enumerate}[leftmargin=*]
\item Dadas las matrices
\[
A=\left(
\begin{array}{rrc}
  4 & 2 & 5 \\
 -1 & 5 & 2 \\
\end{array}
\right) \quad \textrm{y} \quad
B=\left(
\begin{array}{rrc}
 -1 &  4 & 1 \\
  3 & -3 & 0 \\
\end{array}
\right)
\]
se pide:
\begin{enumerate}
\item Calcular $3A-2B$.
\item Calcular $A'B$, $BA'$, $AB'$ y $B'A$.
\item Calcular el determinante de las matrices calculadas en el apartado anterior y calcular su inversa cuando sea posible.
\end{enumerate}




\item  Dadas las matrices
\[
A=\left(
\begin{array}{rrr}
1 & 2 & 0 \\ -1 & 1 & 4
\end{array}
\right) ,\quad
B=\left(
\begin{array}{rrr}
0 & 1 & 0 \\ 3 & -1 & 0
\end{array}
\right) \quad \mbox{y}\quad
C=\left(
\begin{array}{rr}
2 & 1 \\ 3 & -2
\end{array}
\right) , \]
calcular la matriz $X$ tal que $CX+AB'=BB'.$

\item  Dada la matriz
\[
A=\left(
\begin{array}{cccc}
a & 2a-1 & 3 & 1 \\ 1 & 1 & 1 & 1 \\ 0 & -1 & 0 & 1 \\ 3 & 5 & 3 & a
\end{array}
\right) , \]
determinar el rango de $A$ según los valores del parámetro $a$.

\item Considérense siguientes matrices de espín de Pauli. 
\[
A=\left(
\begin{array}{rr}
 1 &  0 \\
 0 & -1 \\
\end{array}
\right)
\quad \textrm{y} \quad
B=\left(
\begin{array}{rr}
 0 & 1 \\
 1 & 0 \\
\end{array}
\right)
\]
Estas matrices se usan en el estudio del espín de electrones en mecánica cuántica. Demostrar que $A^2=I$, $B^2=I$, y $AB=-BA$ (las matrices que cumplen esta última propiedad se llaman anticonmutativas). 

\end{enumerate}
\end{document}
