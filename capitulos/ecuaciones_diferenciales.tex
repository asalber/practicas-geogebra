% Author: Alfredo Sánchez Alberca (asalber@ceu.es)
\chapter{Ecuaciones Diferenciales Ordinarias}

\section{Fundamentos teóricos}

Muchos fenómenos de la naturaleza como la desintegración radiactiva, algunas reacciones químicas, el crecimiento de
poblaciones o algunos problemas gravitatorios responden a determinadas ecuaciones en las que se relaciona una función
con sus derivadas. 
Este tipo de ecuaciones se conocen como \emph{ecuaciones diferenciales} y en esta práctica
estudiaremos cómo resolverlas.

\subsection{Ecuaciones diferenciales ordinarias (E.D.O.)}
Se llama \emph{ecuación diferencial ordinaria (E.D.O.)} a una relación entre una variable independiente $x$, una función
desconocida $y(x)$, y alguna de las derivadas de $y$ con respecto a $x$. 
Esto es, a una expresión de la forma
\[
F(x,y,y',y'',...,y^{(n})=0.
\]

Llamaremos \emph{orden de la ecuación diferencial ordinaria} al mayor orden de las derivadas que aparezcan en la
ecuación. 
Así, la forma más general de una E.D.O. de primer orden es $F(x,y,y')=0$, que puede quedar de la forma $y'=G(x,y)$ si se
puede despejar $y'$.

\subsubsection*{Solución de una E.D.O.}
Diremos que una función $f(x)$ es \emph{solución} o \emph{integral} de la EDO $F(x,y,y',y'',...,y^{(n})=0$, si al
sustituir en ella $y$ y sus derivadas por $f(x)$ y sus derivadas respectivas, la ecuación se satisface, es decir
$F(x,f(x),f'(x),f''(x),...,f^{(n}(x))=0$.

En general una ecuación diferencial admite infinitas soluciones, y se limita su número imponiendo condiciones iniciales.

\subsection{Ecuaciones diferenciales ordinarias de primer orden}
Una ecuación diferencial ordinaria de primer orden es una ecuación de la forma
\[
y'=F(x,y).
\]
Esta es la forma estándar de escribir la ecuación, aunque a veces, también se suele representar en la forma diferencial como
\[
M(x,y)dx+N(x,y)dy=0.
\]

\subsubsection*{Soluciones general y particular de una E.D.O. de primer orden}
Se llama \emph{solución general} o \emph{integral general} de una ecuación diferencial ordinaria de primer orden a una
función $y=f(x,c)$, donde $c$ es una constante real, tal que para cada valor de $c$, la función $y=f(x,c)$ es una
solución de la ecuación diferencial. 
A esta solución así obtenida para un valor concreto de $c$ se le denomina \emph{solución particular} o \emph{integral
particular} de la ecuación diferencial.

En la práctica, la determinación de las constantes que conducen a una solución particular se realiza imponiendo ciertas
condiciones iniciales en el problema, que son los valores que debe tomar la solución en determinados puntos. 
Así, para una ecuación diferencial ordinaria de primer orden $y'=F(x,y)$, una condición inicial se expresaría de la
forma $y(x_{0})=y_{0}$, y la solución particular sería una función $y=f(x)$ tal que $f'(x)=F(x,f(x))$, y $f(x_0)=y_0$.

Por ejemplo, si consideramos la ecuación diferencial $y'=y$, resulta sencillo comprobar que su solución general es
$f(x)=ce^x$, ya que $f'(x)=ce^x$ y se cumple la ecuación. 
Si para esta misma ecuación tenemos la condición inicial $y(0)=1$, entonces, al imponer dicha condición a la solución
general, se tiene $f(0)=ce^0=1$, de donde se deduce que $c=1$, y por tanto la solución particular sería $f(x)=e^x$.

Geométricamente, la solución general de una ecuación diferencial de primer orden representa una familia de curvas,
denominadas \emph{curvas integrales}, una para cada valor concreto asignado a la constante arbitraria. 
En la figura~\ref{g:curvas integrales} se muestran las curvas integrales de la ecuación diferencial $y'=y$.

\begin{figure}[h!]
\begin{center}
\scalebox{1}{\input{img/ecuaciones_diferenciales/curvas_integrales}}
\caption{Familia de curvas integrales que son solución de la ecuación $y'=y$.}
\label{g:curvas integrales}
\end{center}
\end{figure}

\subsubsection*{Existencia y unicidad de soluciones}
El siguiente teorema aporta una condición suficiente, aunque no necesaria, para la existencia y la unicidad de la
solución de una ecuación diferencial ordinaria de primer orden.

\begin{teoremasn}
Si $F(x,y)$ y $\partial F/\partial y$ son funciones continuas en un entorno del punto $(x_0,y_0)$, entonces la ecuación
diferencial $y'=F(x,y)$ tiene una solución $y=f(x)$ que verifica $f(x_0)=y_0$, y además esa solución es única.
\end{teoremasn}
Cuando no se cumplen las condiciones del teorema hay que tener cuidado porque la ecuación puede no tener solución, o
bien tener soluciones múltiples como ocurre con la ecuación $y'=3\sqrt[3]{y^2}$, que tiene dos soluciones $y=0$ y
$y=x^3$ que pasan por el punto $(0,0)$, ya que $\frac{\partial}{\partial y}(3\sqrt[3]{y^2})=2/\sqrt[3]{y}$ que no existe
en $(0,0)$.

Desafortunadamente, el teorema anterior sólo nos habla de la existencia de una solución pero no nos proporciona la forma
de obtenerla. 
En general, no existe una única técnica de resolución de ecuaciones diferenciales ordinarias de primer orden
$M(x,y)dx+N(x,y)dy=0$, sino que dependiendo de la forma que tengan $M(x,y)$ y $N(x,y)$, se utilizan distintas técnicas.


\subsection{EDO de variables separables}
Una E.D.O. de primer orden es de \emph{variables separables} si se puede poner de la forma $y'g(y)=f(x)$ o bien
$M(x)dx+N(y)dy=0$, donde $M(x)$ es una función que sólo depende de $x$ y $N(y)$ sólo depende de $y$.

La solución de una ecuación de este tipo se obtiene fácilmente integrando $M(x)$ y $N(y)$ por separado, es decir
\[
\int M(x)\,dx=-\int N(y)\,dy.
\]

\subsection{EDO Homogéneas}
Se dice que una función $F(x,y)$ es \emph{homogénea} de grado $n$ si se cumple $F(kx,ky)=k^nF(x,y)$.

Una E.D.O. de primer orden es \emph{homogénea} si se puede poner de la forma
$y'=f\left(\dfrac{y}{x}\right)$ o bien $M(x,y)dx+N(x,y)dy=0$ donde $M(x,y)$ y $N(x,y)$ son funciones homogéneas del mismo grado.

Las ecuaciones homogéneas son fácilmente reducibles a ecuaciones de variables separables mediante el cambio $y=ux$,
siendo $u$ una función derivable de $x$.

\subsection{EDO Lineales}
Una E.D.O. de primer orden es \emph{lineal} si se puede poner de la forma $y'+ P(x)y = Q(x)$, donde $P$ y $Q$ son
funciones continuas de $x$.

Para resolver este tipo de ecuaciones se utiliza la técnica de los factores integrantes. 
Un factor integrante es una función $u(x)$ cuya derivada sea $P(x)u(x)$, con lo que al multiplicar $u(x)$ por el lado
izquierdo de la ecuación, el resultado es la derivada del producto $u(x)y$, es decir
\[
u(x)y'+u(x)P(x)y=\frac{d}{dx}(u(x)y).
\]
A partir de aquí, si también multiplicamos por $u(x)$ el lado derecho de la ecuación tenemos
\[
\frac{d}{dx}(u(x)y)=Q(x)u(x),
\]
por lo que integrando, resulta
\[
u(x)y=\int Q(x)u(x)\,dx
\]
de donde se puede despejar fácilmente $y$.

Por último, resulta fácil comprobar que un factor integrante de esta ecuación es $u(x)=e^{\int P(x)\, dx}$, de manera
que la solución quedaría
\[
y=e^{-\int P(x)\,dx}\int Q(x)e^{\int P(x)\,dx}\,dx+C.
\]

\newpage

\section{Ejercicios resueltos}
\begin{indicacion}
Para resolver ecuaciones diferenciales ordinarias de primer orden utilizaremos los comandos
\begin{quote}
\texttt{DSOLVE1\_GEN(p,q,x,y,c)} proporciona la solución general de $p(x,y)+q(x,y)y'=0$.

\texttt{DSOLVE1(p,q,x,y,$x_{0},y_{0}$)} proporciona la solución particular de $p(x,y)+q(x,y)y'=0$, con la condición inicial $y_{0}=y(x_{0})$.
\end{quote}
\end{indicacion}

\begin{enumerate}[leftmargin=*]
\item Resolver las siguientes ecuaciones diferenciales de variables separables y dibujar sus curvas integrales para las
constantes de integración $c=-1$, $c=-2$ y $c=-3$:
\begin{enumerate}
\item $-2x(1+e^y)+e^y(1+x^{2})y'=0$.
\begin{indicacion}
Para resolver la ecuación diferencial observamos que ya está escrita en la forma $p(x,y)+q(x,y)y'=0$, con $p(x,y)=-2x(1+e^y)$ y
$q(x,y)=e^y(1+x^{2})$.
\begin{enumerate}
\item Introducir la expresión \comando{DSOLVE1\_GEN(-2x(1+\#e\^{}y),\#e\^{}y(1+x\^{}2),x,y,c)}.
\item Hacer clic en el botón \boton{Simplificar}.
\end{enumerate}
Para dibujar las curvas integrales:
\begin{enumerate}[resume]
\item Hacer clic en el botón \boton{Sustituir}, seleccionar la variable $c$, introducir el valor -1 y hacer clic en el
botón \boton{Simplificar}.
\item Hacer clic en el botón \boton{Ventana 2D} para pasar a la ventana gráfica 2D.
\item Hacer clic en el botón \boton{Representar expresión}.
\item Repetir el mismo procedimiento introduciendo los valores -2 y -3 para el valor de $c$.
\end{enumerate}
\end{indicacion}

\item $y-xy'=1+x^2y'$.
\begin{indicacion}
Para resolver la ecuación diferencial primero hay que ponerla en la forma $p(x,y)+q(x,y)y'=0$,
\[
y-xy'=1+x^2y' \Leftrightarrow 1+x^2y'-y+xy'=0 \Leftrightarrow 1-y+(x^2+x)y'=0
\]
de manera que $p(x,y)=1-y$ y $q(x,y)=x^2+x$.
\begin{enumerate}
\item Introducir la expresión \comando{DSOLVE1\_GEN(1-y,x\^{}2+x,x,y,c)}.
\item Hacer clic en el botón \boton{Simplificar}.
\end{enumerate}
Para dibujar las curvas integrales:
\begin{enumerate}[resume]
\item Hacer clic en el botón \boton{Sustituir}, seleccionar la variable $c$, introducir el valor -1 y hacer clic en el
botón \boton{Simplificar}.
\item Hacer clic en el botón \boton{Ventana 2D} para pasar a la ventana gráfica 2D.
\item Hacer clic en el botón \boton{Representar expresión}.
\item Repetir el mismo procedimiento introduciendo los valores -2 y -3 para el valor de $c$.
\end{enumerate}
\end{indicacion}
\end{enumerate}


\item Resolver las siguientes ecuaciones diferenciales con las condiciones iniciales dadas.
\begin{enumerate}
\item $x\sqrt{1-y^2}+y\sqrt{1-x^2} y'=0$, con la condición inicial $y(0)=1$.
\begin{indicacion}
Para resolver la ecuación diferencial observamos que ya está escrita en la forma $p(x,y)+q(x,y)y'=0$, con
$p(x,y)=x\sqrt{1-y^2}$ y $q(x,y)=y\sqrt{1-x^2}$.
\begin{enumerate}
\item Introducir la expresión \comando{DSOLVE1(xsqrt(1-y\^{}2),ysqrt(1-x\^{}2),x,y,0,1)}.
\item Hacer clic en el botón \boton{Simplificar}.
\end{enumerate}
\end{indicacion}


\item $(1+e^x)yy'=e^y$, con la condición inicial $y(0)=0$.
\begin{indicacion}
Para resolver la ecuación diferencial primero hay que ponerla en la forma $p(x,y)+q(x,y)y'=0$,
\[
(1+e^x)yy'=e^y \Leftrightarrow -e^y+(1+e^x)yy'=0,
\]
de manera que $p(x,y)=-e^y$ y $q(x,y)=(1+e^x)y$.
\begin{enumerate}
\item Introducir la expresión \comando{DSOLVE1(-\#e\^{}y,(1+\#e\^{}x)y,x,y,0,0)}.
\item Hacer clic en el botón \boton{Simplificar}.
\end{enumerate}
\end{indicacion}


\item $y'+y\cos x=\sen x\cos x$ con la condición inicial $y(0)=1$.
\begin{indicacion}
Para resolver la ecuación diferencial primero hay que ponerla en la forma $p(x,y)+q(x,y)y'=0$,
\[
y'+y\cos x=\sen x\cos x \Leftrightarrow -\sen x\cos x+y\cos x+y'=0,
\]
de manera que $p(x,y)=-\sen x\cos x+y\cos x$ y $q(x,y)=1$.
\begin{enumerate}
\item Introducir la expresión \comando{DSOLVE1(-sinxcosx+ycosx,1,x,y,0,1)}.
\item Hacer clic en el botón \boton{Simplificar}.
\end{enumerate}
\end{indicacion}
\end{enumerate}

\item El azúcar se disuelve en el agua con una velocidad proporcional a la cantidad que queda por disolver.
Si inicialmente había $13.6$ kg de azúcar y al cabo de $4$ horas quedan sin disolver $4.5$ kg, ¿cuánto tardará en disolverse el $95\%$ del
azúcar contando desde el instante inicial?
\begin{indicacion}
La ecuación diferencial que explica la disolución del azúcar es $y'=ky$, donde $y$ es la cantidad de azúcar que queda
por disolver, $x$ es el tiempo y $k$ es la constante de disolución del azúcar. 
Para resolver la ecuación diferencial primero hay que ponerla en la forma $p(x,y)+q(x,y)y'=0$,
\[
y'=ky \Leftrightarrow -ky+y'=0,
\]
de manera que $p(x,y)=-ky$ y $q(x,y)=1$.
\begin{enumerate}
\item Introducir la expresión \comando{DSOLVE1(-ky,1,x,y,0,13.6)}.
\item Hacer clic en el botón \boton{Simplificar}.
\end{enumerate}
Para obtener la constante de disolución se impone la otra condición $y(4)=4.5$:
\begin{enumerate}[resume]
\item Hacer clic en el botón \boton{Sustituir}, seleccionar la variable $x$ e introducir el valor $4$, seleccionar la
variable $y$ e introducir el valor $4.5$, y hacer clic en el botón \boton{Simplificar} y después en el botón
\boton{Aproximar}.
\end{enumerate}
Finalmente, para obtener el tiempo que tiene que pasar hasta que quede un $5\%$ del azúcar inical, es decir $13.6\cdot
0.05=0.68$ kg:
\begin{enumerate}[resume]
\item Seleccionar la expresión correspondiente a la solución general de la ecuación diferencial.
\item Hacer clic en el botón \boton{Sustituir}, seleccionar la variable $k$ e introducir el valor de la constante calculado antes,
seleccionar la variable $y$ e introducir el valor $0.68$, y hacer clic en el botón \boton{Simplificar} y después en el botón
\boton{Aproximar}.
\end{enumerate}
\end{indicacion}

\end{enumerate}


\section{Ejercicios propuestos}
\begin{enumerate}[leftmargin=*]

\item Resolver las siguientes ecuaciones diferenciales:
\begin{enumerate}
\item $(1+y^{2})+xyy'=0$.
\item $xy'-4y+2x^2+4=0$.
\item $(y^{2}+xy^{2})y'+x^{2}-yx^{2}=0$.
\item $(x^3-y^3)dx+2x^2ydy=0$.
\item $(x^2+y^2+x)+xydy=0$.
\end{enumerate}

\item Hallar las curvas tales que en cada punto $(x,y)$ la pendiente de la recta tangente sea igual al cubo de la
abscisa en dicho punto. 
¿Cuál de estas curvas pasa por el origen?

\item Al introducir glucosa por vía intravenosa a velocidad constante, el cambio de concentración global de glucosa con
respecto al tiempo $c(t)$ se explica mediante la siguiente ecuación diferencial
\[
\frac{dc}{dt}=\frac{G}{100V}-kc,
\]
donde $G$ es la velocidad constante a la que se suministra la glucosa, $V$ es el volumen total de la sangre en el cuerpo y $k$ es una constante positiva que depende del paciente. Se pide calcular $c(t)$.

\item En una reacción química, un cierto compuesto se transforma en otra sustancia a un ritmo proporcional a la cantidad
no transformada. 
Si había inicialmente 100 gr de sustancia original y 60 gr tras una hora, ¿cuanto tiempo pasará hasta
que se haya transformado el 80\% del compuesto?
\end{enumerate}

